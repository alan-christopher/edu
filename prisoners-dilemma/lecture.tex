\PassOptionsToPackage{table}{xcolor}
\documentclass[pdf]{beamer}
\mode<presentation>{\usetheme{Dresden}}
\usepackage{lmodern}
\usepackage[normalem]{ulem}
\usepackage{amsmath,textcomp,amssymb,geometry,graphicx,listings,array,color,amsthm}

\usepackage[noend]{algpseudocode}

\usepackage{tikz}
\usepackage{multicol}
\usetikzlibrary{shapes,snakes}
\usetikzlibrary{positioning}
\usetikzlibrary{arrows}
\usetikzlibrary{fit}
\usetikzlibrary{math}

%% For on-slide alerting of nodes
\tikzstyle{alert} = [text=black, fill=blue!20, draw=black]
\setbeamercolor{alerted text}{fg=blue}
\tikzset{alerton/.code args={<#1>}{%
  \only<#1>{\pgfkeysalso{alert}} % \pgfkeysalso doesn't change the path
}}

%% utils for removing uninteresting sections from the navbar
%% https://tex.stackexchange.com/questions/317774/hide-section-from-sidebar
\makeatletter
\let\beamer@writeslidentry@miniframeson=\beamer@writeslidentry%
\def\beamer@writeslidentry@miniframesoff{%
  \expandafter\beamer@ifempty\expandafter{\beamer@framestartpage}{}% does not happen normally
  {%else
    % removed \addtocontents commands
    \clearpage\beamer@notesactions%
  }
}
\newcommand*{\miniframeson}{\let\beamer@writeslidentry=\beamer@writeslidentry@miniframeson}
\newcommand*{\miniframesoff}{\let\beamer@writeslidentry=\beamer@writeslidentry@miniframesoff}
\makeatother

%% preamble
\title{Prisoner's Dilemma}
\subtitle{Theme and Variations}
\author{A.C.}
\date{\today}

\AtBeginSection[]
{
  \miniframesoff
  \begin{frame}{Outline}
    \tableofcontents[currentsection,hideothersubsections]
  \end{frame}
  \miniframeson
}

\definecolor{darkred}{rgb}{0.7,0,0}
\definecolor{darkgreen}{rgb}{0,0.5,0}
\definecolor{darkblue}{rgb}{0,0,0.5}
\definecolor{darkpurple}{rgb}{0.4, 0.0, 0.4}

%% Code font settings
\lstset{
  showstringspaces=false,
  basicstyle=\scriptsize\ttfamily,
  commentstyle=\color{darkred},
  stringstyle=\color{darkgreen},
  keywordstyle=\bfseries\color{darkpurple},
}

%%%%%%%%%%%%%%%%%%%%%%%%%%
% Start of Actual slides %
%%%%%%%%%%%%%%%%%%%%%%%%%%
\begin{document}
\begin{frame}
  \titlepage
\end{frame}

\section{The Prisoner's Dilemma}
\subsection{Single Round}
\begin{frame}{Nash Equilibrium}
  We say that players in a non-cooperative game are in a \textbf{Nash Equilibrium}
  if neither can improve their score by adopting a new strategy.
  \begin{itemize}
  \pause\item Players \emph{can} communicate. Their strategies are assumed transparent
    to each other.
  \pause\item Players \emph{cannot} cooperate. It's still a Nash Equilibrium if both
    players changing strategy together yields improvement.
  \end{itemize}
\end{frame}

\begin{frame}{Prisoner's Dilemma}
  % TODO: insert a 2x2 showing the Axelrod reward/punish matrix

  % Notes:
  % - Stage game
  % - Optimal play: Defect
  % - Nash Equilibrium
  % - Opponent-independent
\end{frame}

\begin{frame}{Homo Economicus}
  % TODO: insert am I the one that is so out of touch? meme but with models for
    % out of touch and humans for are-wrong
\end{frame}

\subsection{Fixed, Finite Rounds}

\begin{frame}{Iterated Prisoner's Dilemma}

  What happens if we play several rounds of prisoner's dilemma? A tension
  between developing good will (cooperating) and spending it (defecting)?

  Simple rules:
  \begin{itemize}
  \item $N$ rounds of play.
  \item Same rewards in every round.
  \item Final score is sum of individual rounds.
  \item Every player has perfect recall.
  \end{itemize}
\end{frame}

\begin{frame}{Rational Play}
  \begin{itemize}
  \item Final round reduces to plain PD;
    \begin{itemize}
      \item D-D
    \end{itemize}
  \pause\item Penultimate round reduces to plain PD, followed by D-D.
    \begin{itemize}
      \item D-D, D-D
    \end{itemize}
  \pause\item round before that reduces to plain PD, followed by D-D, D-D.
    \begin{itemize}
      \item D-D, D-D, D-D
    \end{itemize}
  \pause\item $\ldots$
  \pause\item $\ldots$
  \pause\item D-D,D-D,D-D,$\ldots$,D-D
\end{itemize}
\end{frame}

\begin{frame}{Greater Fools}
  \begin{itemize}
  \item Iterated PD still yields universal defection under optimal, rational play.
  \pause\item Irrational partners could cooperate.
  \begin{itemize}
    \item We might want to (sometimes) cooperate as well, depending on partner.
    \end{itemize}
  \end{itemize}
\end{frame}


\subsection{Indefinite Rounds}

\section{Axelrod's Tournament}


\section{Theories of Single-Round Cooperation}
\begin{frame}{Super-Rationality}
\end{frame}

\begin{frame}{Mutual Interest}
\end{frame}

\end{document}
